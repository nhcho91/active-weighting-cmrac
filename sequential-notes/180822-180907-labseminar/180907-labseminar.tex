\documentclass[]{beamer}

\title{Data-Efficient Online Optimization for Dynamical Systems Without
Persistent Excitation}
\author{Seong-hun Kim, Namhoon Cho, Youdan Kim}
\institute{
	Flight Dynamics and Control Lab. \\
	Seoul Natl.~Univ. \\
}
\date{Lab Seminar \\ September 7, 2018}


% ~~~~~~~
\usenavigationsymbolstemplate{}
\setbeamertemplate{footline}[frame number]
\setbeamerfont{footnote}{size=\tiny}

% ~~~~~~~
\usepackage[backend=bibtex, style=ieee, citetracker=true]{biblatex}
\bibliography{../global}

\usepackage{mathtools}
\usepackage{physics}

\theoremstyle{plain}
\newtheorem{proposition}{Proposition}

\theoremstyle{definition}
\newtheorem{assumption}{Assumption}
\newtheorem{algorithm}{Algorithm}

\theoremstyle{remark}
\newtheorem{observation}{Observation}
\newtheorem{remark}{Remark}

\usepackage{multimedia} % for Linux, Mac (use .mov)
\usepackage{media9} % for Windows (use .mp4)

\usepackage{algorithm, algpseudocode}

% ~~~~~~~
\newcommand{\MC}[1]{\mathcal{#1}}
\newcommand{\MB}[1]{\mathbb{#1}}
\DeclareMathOperator{\gap}{gap}
\DeclareMathOperator{\Gap}{Gap}
\DeclareMathOperator{\diag}{diag}
\newcommand{\lmin}{\lambda_{\min}}
\newcommand{\lmax}{\lambda_{\max}}

% ~~~~~~~~~~~~~~
\begin{document}

\begin{frame}
	\maketitle
\end{frame}


\section{Introduction}
% ====================

\begin{frame}{Composite Adaptive Control}
	\begin{description}
		\item[CAC]: adaptive control + online parameter estimation
	\end{description}

	\vspace{1em}
	Slotine said~\footfullcite{slotine_applied_1991}
	\begin{itemize}
		\item {\it When there is parameter uncertainty in a dynamic system (linear or
			nonlinear), one way to reduce it is to use parameter estimation,}
		\item {\it i.e., inferring the values of the parameters from the
			measurements of input and output signals of the system.}
	\end{itemize}
\end{frame}
% ~~~~~~~~~

\begin{frame}{Example: Online Parameter Estimation}
	Consider the following scalar system with $w^\ast \in \MB{R}^N$
	\begin{equation*}
		\dot{x} = f(x) + g(x) \qty(u(x) + {w^\ast}^T \phi(x)).
	\end{equation*}
	\pause
	% ~~~~
	Now, suppose that we can measure $x$ and $\dot{x}$, and obviously $u(x)$ and
	$\phi(x)$. Also, assume that $g(x)$ is non-zero for all time. Then,
	\begin{equation*}
		\phi(x)^T w^\ast = g^{-1}(x) \qty(\dot{x} - f(x)) - u(x) \eqqcolon y.
	\end{equation*}
	\pause
	% ~~~~
	\begin{observation}
		\begin{enumerate}
			\item This equation has the same form of linear regression.
			\item If $w^\ast$ is constant, and the measurements are perfect, then we
				only need $N$ independent data of $(x, y)$ to determine the parameter
				$w^\ast$.
		\end{enumerate}
	\end{observation}
\end{frame}
% ~~~~~~~~~


\section{Problem Formulation}
% ===========================

\begin{frame}{Dataset}
	Consider a sequence of data
	\begin{equation*}
		\MC{D}_0,\ \MC{D}_1,\ \MC{D}_2,\ \dotsc,\ \MC{D}_{k-2},\
		\MC{D}_{k-1},\ \MC{D}_k
	\end{equation*}
	where $\MC{D}_i \coloneqq \qty(x(t_i),\ y(t_i))$. Also, we can write
	\begin{align*}
		\Phi_{\{i_1,\dotsc,i_M\}}
		&\coloneqq \mqty[\phi(x(t_{i_1})) & \cdots & \phi(x(t_{i_M}))]
		\mqty[\phi^T(x(t_{i_1})) \\ \vdots \\ \phi^T(x(t_{i_M}))] \\
		&= \sum_{j=1}^M \phi(x(t_{i_j})) \phi^T(x(t_{i_j})).
	\end{align*}

	\begin{assumption}
		There exist at least $M \ge N$ data such that $\Phi_{\{i_1,\dotsc,i_M\}}$
		has full rank.
	\end{assumption}
\end{frame}
% ~~~~~~~~~

\begin{frame}{Problem Classification}
	\begin{enumerate}
		\item<+-> \textbf{Perfect data, constant parameter:}
			\begin{itemize} 
				\item Choose \textit{any} $M \ge N$ independent data so that the rank
					is full.
				\item \textit{Optimal strategy}: Choose \alert{best combination} among
					all data, which requires a huge memory.
			\end{itemize}
		\item<+-> \textbf{Perfect data, varying parameter:}
			\begin{itemize}
				\item Choose $M \ge N$ independent data measured \textit{recently}.
				\item \textit{Concurrent learning}: \alert{LIFO} if last one increases
					the rank (or the smallest singular value). \alert{However}, it cannot
					update until the condition is satisfied.
				\item \textit{Optimal strategy}: Similar with more weighting to recent
					data.
			\end{itemize}
		\item<+-> \textbf{Imperfect data, varying parameter:}
			\begin{itemize}
				\item \textit{(mini) Batch}: Choose \alert{$M \gg N$} data with
					forgetting factor.
				\item \textit{Iterative algorithm}: Rank-one update with
					forgetting factor. 
		\end{itemize}
	\end{enumerate}
	\uncover<+->{\it A limitation for \alert{dynamical systems}: Most of
		iterative algorithms prove its stability based on i.i.d.~sampling
	assumption.\footfullcite{nemirovski_robust_2009}}
\end{frame}
% ~~~~~~~~~

\begin{frame}{Trajectory of $\phi(x)$}
	\centering
	% \movie[loop, poster, showcontrols, width=\textwidth]{
	% 	\includegraphics[width=\textwidth]{media/phiden.jpg}
	% }{media/phiden.mov}
	\includemedia[
		width=\textwidth,
		passcontext,  %show VPlayer's right-click menu
		addresource=media/phiden.mp4,
		flashvars={
			source=media/phiden.mp4
		}
	]{\fbox{Click!}}{VPlayer.swf}
\end{frame}
% ~~~~~~~~~


\section{Generalized Estimator}
% =============================

\begin{frame}{Generalization of Previous Works}
	\begin{proposition}
		The gradient estimators, the standard least-squares
		estimators\footfullcite{slotine_applied_1991}, concurrent-learning
		estimators\footfullcite{chowdhary_exponential_2014}, and many other recent
		integration-based estimators\footfullcite{cho_composite_2018}$^,$
		\footfullcite{pan_composite_2018} can be representative by the following
		general form
		\begin{gather*}
			\dot{W}(t) = \Gamma_1 \phi(t) e^T P B \alert{- \int_0^t c(t, \tau)
			\phi(\tau) \epsilon^T(t, \tau) \dd{\tau}}, \\
			c(t,\tau) = \beta(\tau) \exp(- \int_\tau^t \alpha(\nu) \dd{\nu}).
		\end{gather*}
	\end{proposition}
\end{frame}
% ~~~~~~~~~

\begin{frame}{Example: The Gradient and SLS Estimators}
	Let us define
	\begin{equation*}
		\alpha(\nu) = \frac{1}{\Delta} \qc \beta(\tau) = \frac{1}{\Delta}
		P(\tau).
	\end{equation*}
	\pause
	% ~~~~
	Then,
	\begin{equation*}
		c(t, \tau) \to P(\tau) \cdot \delta(t-\tau) \qas \Delta \to 0,
	\end{equation*}
	where $\delta: [0, t] \to \MB{R}_{+}$ is the Dirac delta function defined on
	$[0, t]$.
	\pause
	% ~~~~

	This leads the general form to the gradient and standard least-squares
	estimators
	\begin{equation*}
		\dot{W}(t) = \Gamma_1 \phi(t) e^T P B \alert{- P(t) \phi(t) \epsilon^T(t)}.
	\end{equation*}

	\textbf{Note that} these estimators require the \alert{Persistent Excitation
	(PE)} conditions, which is
	\begin{equation*}
		\int_t^{t + T} \phi(\tau) \phi^T(\tau) \dd{\tau} \succeq \gamma_1 I_N \qc
		\text{for some } \gamma_1 > 0,\ T > 0,\ \text{for all } t \ge 0
	\end{equation*}
\end{frame}
% ~~~~~~~~~

\begin{frame}{Example: Concurrent-Learning Estimators}
	With the same definitions of $\alpha$ and $\beta$
	\begin{equation*}
		\alpha(\nu) = \frac{1}{\Delta} \qc \beta(\tau) = \frac{\Gamma_2}{\Delta},
	\end{equation*}
	let us expand the definition of $c(t, \tau)$ a little bit as
	\begin{equation*}
		c^\prime(t, \tau) = \sum_{\substack{j=1 \\ t_{i_j} \in [0, t]}}^{M}
		c(t_{i_j}, \tau) = \sum_{\substack{j=1 \\ t_{i_j} \in [0, t]}}^{M}
		\Gamma_2 \delta(t_{i_j} - \tau),
	\end{equation*}
	Then, we have the concurrent-learning estimator
	\begin{equation*}
		\dot{W}(t) = \Gamma_1 \phi(t) e^T P B \alert{- \Gamma_2
		\textstyle\sum_j \phi_{i_j} \epsilon_{i_j}^T(t)}.
	\end{equation*}
	\pause
	% ~~~~

	\textbf{Note that} this estimators require the \alert{rank condition} which
	reads
	\begin{equation*}
		\textstyle\sum_j \phi_{i_j} \phi^T_{i_j} \succeq \gamma_1 I_N \qc \text{for
		some } \gamma_1 > 0,\ \{i_j\}.
	\end{equation*}
\end{frame}
% ~~~~~~~~~

\begin{frame}{Example: Integration-Based Estimators}
	This is obvious, as we can set $\beta(\tau) = 1$, which leads
	\begin{equation*}
		\dot{W}(t) = \Gamma_1 \phi(t) e^T P B \alert{- \int_0^t
			\exp{- \int_\tau^t \alpha(\nu) \dd{\nu}} \phi(\tau) \epsilon^T(t,
		\tau) \dd{\tau}}.
	\end{equation*}

	\textbf{Note that} N. Cho suggested the more relaxed condition called
	\alert{Finite Excitation (FE)} condition, which reads
	\begin{equation*}
		\int_{t_s}^{t_s + T} \phi(\tau) \phi^T(\tau) \dd{\tau} \succeq \gamma_1
		I_N,
	\end{equation*}
	for some $\gamma_1 > 0,\ T > 0$ and for \alert{some} $t_s \ge 0$.
\end{frame}
% ~~~~~~~~~

\begin{frame}{So, Why the General Form is Useful?}
	Let
	\begin{equation*}
		\MC{A}(t) = \int_0^t c(t, \tau) \phi(\tau) \phi^T(\tau) \dd{\tau}.
	\end{equation*}
	\textbf{Note that} we may define a \alert{generalized} excitation condition
	\begin{equation*}
		\MC{A}(t) \succeq \gamma_1 I_N \qc \text{for \alert{some} } \alert{c(t,
		\tau)},\ \text{for all } t \ge 0.
	\end{equation*}
	\pause
	Recall that
	$ c(t, \tau) = \beta(\tau) \exp(- \int_\tau^t \alpha(\nu) \dd{\nu}), $
	for $\alpha: \MB{R}_{+} \to \MB{R}$ and $\beta: \MB{R}_{+} \to \MB{R}$,
	which leads to 
	\begin{equation*}
		\dot{\MC{A}}(t) = - \alpha(t) \MC{A}(t) + \beta(t) \phi(t) \phi^T(t) \qc
		\MC{A}(0) = 0,
	\end{equation*}
	and its discrete counter part
	\begin{equation*}
		A^{k+1} = \alert{a_k} A^k + \alert{b_k} \phi_k \phi_k^T \qc A^0 = 0.
	\end{equation*}
\end{frame}
% ~~~~~~~~~


\section{Data-Efficient Rank-One Update}
% ======================================

\begin{frame}{The Rank-One Update}
	Rewrite the discrete counter part as
	\begin{equation*}
		A^\prime = \alert{a} A + \alert{b} v v^T,
	\end{equation*}
	which can be viewed as a modified rank-one
	update,\footfullcite{bunch_rank-one_1978} where the numerical mathematics
	society is interested in the case of $a_k = 1$.
	\pause

	\vspace{1em}
	\textbf{Ipsen and Nadler} introduced\footfullcite{ipsen_refined_2009} the
	refined perturbation bounds for eigenvalues of $A^{k+1}$.

	\vspace{1em}
	Let us order the eigenvalues of a real symmetric matrix algebraically as
	\begin{equation*}
		\lambda_{\max} = \lambda_n \ge \lambda_{n-1} \ge \cdots \ge \lambda_2 \ge
		\lambda_1 = \lambda_{\min}
	\end{equation*}
\end{frame}
% ~~~~~~~~~

\begin{frame}
	\begin{theorem}[Ipsen, see~\cite{ipsen_refined_2009}]
		\label{thm:ipsen}
		Let $A \in \MB{C}^{n \times n}$ be Hermitian, $v \in \MB{C}^n$.
		\begin{enumerate}[(i)]
			\item<+-> (smallest eigenvalue). Let
				\begin{equation*}
					L_{\pm} \coloneqq \mqty(\lambda_2(A) & 0 \\ 0 & \lambda_1(A)) \pm
					\mqty(\norm{v_{2:n}} \\ v_1) \mqty(\norm{v_{2:n}} & \bar{v}_1).
				\end{equation*}
				Then
				\begin{equation*}
					\lmin(L_{\pm}) \le \lmin(A \pm v v^\ast).
				\end{equation*}
			\item<+-> (largest eigenvalue). Let
				\begin{equation*}
					U_{\pm} \coloneqq \mqty(\lambda_n(A) & 0 \\ 0 & \lambda_{n-1}(A)) \pm
					\mqty(v_n \\ \norm{v_{1:n-1}}) \mqty(\bar{v}_n & \norm{v_{1:n-1}}).
				\end{equation*}
				Then
				\begin{equation*}
					\lmax(A \pm v v^\ast) \le \lmax(U_{\pm}).
				\end{equation*}
		\end{enumerate}
	\end{theorem}
\end{frame}
% ~~~~~~~~~

\begin{frame}{Main Result}
	Given $A \in \MB{S}^{n \times n}$, and $v \in \MB{R}^n$, let 
	\begin{align*}
		f_{i,j}^c(a, b) &\coloneqq \lambda_i \lambda_j a^2 + \qty(\norm{v_{-i}}^2
		\lambda_i + \abs{v_i}^2 \lambda_j) a b \\
		&\quad - \qty(\lambda_i + \lambda_j) a c - \norm{v}^2 b c + c^2,\\
		g_{i,j}^c(a, b) &\coloneqq \qty(\lambda_1 + \lambda_2) a + \norm{v}^2 b -
		2 c.
	\end{align*}
	\pause
	% ~~~~

	\begin{theorem}
		Suppose that there exist $a, b \in \MB{R}$ and $c \in \MB{R}_{+}$, such
		that 
		\begin{enumerate}[(i)]
			\item $f_{1, 2}^c(a,b) \ge 0$ and $g_{1, 2}^c(a,b) \ge 0$, then,
				\begin{equation*}
					\lambda_{\min} \qty(a A + b v v^T) \ge c.
				\end{equation*}
			\item $f_{n, n-1}^c(a,b) \ge 0$ and $g_{n, n-1}^c(a,b) \le 0$, then,
				\begin{equation*}
					\lambda_{\max} \qty(a A + b v v^T) \le c.
				\end{equation*}
		\end{enumerate}
	\end{theorem}
\end{frame}
% ~~~~~~~~~

\begin{frame}{Following Lemmas}
	\begin{lemma}
		Let $A \in \MB{S}^{n \times n}$ and $v \in \MB{R}^n$. Then, for all $c_1,
		c_n \in \MB{R}$ such that $c_1/\lambda_1 \le c_n/\lambda_n$, there exist
		$a>0, b \in \MB{R}$ satisfying
		\begin{equation*}
			c_1 \le \lambda_i(a A + b v v^T) \le c_2,
		\end{equation*}
		for $i = 1, \dotsc, n$.
	\end{lemma}
	\pause
	% ~~~~

	\begin{lemma}
		Let $A \in \MB{S}^{n \times n}$ and $v \in \MB{R}^n$. Suppose that
		${\abs{v_1}^2}/{\lambda_1} \ne {\abs{v_n}^2}/{\lambda_n}$,
		$\lambda_2 > \lambda_1$, and $\lambda_n > \lambda_{n-1}$. Then, for all
		$c_1, c_2$ such that $c_1/\lambda_1 \le c_n/\lambda_n$, there exist $a > 0$
		and $b \in \MB{R}$ satisfying
		\begin{equation*}
			c_1 < \lambda_i(a A + b v v^T) < c_n,
		\end{equation*}
		for all $i = 1, \dotsc, n$.
	\end{lemma}
\end{frame}
% ~~~~~~~~~

\begin{frame}{Illustrative Example}
	\begin{figure}[h]
		\includegraphics[width=\textwidth]{media/ba_region_color.png}
	\end{figure}
\end{frame}
% ~~~~~~~~~

\begin{frame}{Algorithm}
	\begin{algorithm}[H]
		\begin{algorithmic}[1]
			\caption{Greedy maximizing the smallest eigenvalue}
			\Require $\bar{\lambda} \ge \lambda_{\max}$
			\State $S \gets \{ s_j \in \{\lambda_i\} | s_j > 0, s_1 \ge 0,\ \text{for
			} j \ge 2 \}$
			\State Parameterize $r \gets r(a, b)$ such that
			\begin{equation*}
				f_{n, n-1}^{\bar{\lambda}}(a, b) = 0 \qc
				g_{n, n-1}^{\bar{\lambda}}(a, b) \le 0
			\end{equation*}
			\ForAll{$r \in [r_{\min}, r_{\max}]$}
			\State $a \gets a(r)$, $b \gets b(r)$
			\State Line search to maximize $c$ sarisfying
			\begin{equation*}
				f_{1, 2}^c (a, b) = 0 \qc
				g_{1, 2}^c (a, b) \ge 0
			\end{equation*}
			\EndFor             
		\end{algorithmic}
	\end{algorithm}
\end{frame}
% ~~~~~~~~~


\section{Simulation Results}
% ==========================

\begin{frame}{Simulation Setup}
	We consider the following system
	\begin{equation*}
		\dot{x}(t) = \mqty[0 & 1 & 0 \\ -15.8 & -5.6 & -17.3 \\ 1 & 0 & 0] x(t) +
		\mqty[0 \\ 0 \\ -1] c(t) + \mqty[0 \\ 1 \\ 0] \qty(u(t) + \Delta(x)),
	\end{equation*}
	where $\Delta(x) = {W^\ast}^T \phi(x)$, for
	\begin{equation*}
		W^\ast \coloneqq [-18.59521, 15.162375, -62.45153, 9.54708, 21.45291]^T,
	\end{equation*}
	and
	\begin{equation*}
		\phi(x) \coloneqq [x_1, x_2, \abs{x_1} x_2, \abs{x_2} x_2, x_1^3]^T.
	\end{equation*}
	The command signal $c(t)$ is given by a square signal, and the reference
	model is
	\begin{equation*}
		\dot{x}_r(t) = \mqty[0 & 1 & 0 \\ -15.8 & -5.6 & -17.3 \\ 1 & 0 & 0]
		x_r(t) + \mqty[0 \\ 0 \\ -1] c(t).
	\end{equation*}
\end{frame}
% ~~~~~~~~~

\begin{frame}{States and Input}
	\makebox[\textwidth][c]{
		\includegraphics[width=1.2\textwidth]{media/state_input.png}
	}
\end{frame}
% ~~~~~~~~~

\begin{frame}{Parameter Estimation}
	\makebox[\textwidth][c]{
		\includegraphics[width=0.8\textwidth]{media/parameter_estimation.png}
	}
\end{frame}
% ~~~~~~~~~

\begin{frame}{Eigenvalues}
	\makebox[\textwidth][c]{
		\includegraphics[height=0.9\textheight]{media/eigenvalues.png}
	}
\end{frame}
% ~~~~~~~~~

\begin{frame}{$a$ and $b$}
	\makebox[\textwidth][c]{
		\includegraphics[width=1.2\textwidth]{media/a_and_b.png}
	}
\end{frame}
% ~~~~~~~~~


\section{Conclusion}
% ==================

\begin{frame}{Conclusion}
	\begin{itemize}
		\item We generalized recent composite adaptive update laws.
		\item We suggested the systematic method to manipulate the generalize
			update law based on the concept of maximizing the minimum eigenvalue.
		\item We provided a sufficient condition to guarantee the update law.
	\end{itemize}
\end{frame}

\end{document}

